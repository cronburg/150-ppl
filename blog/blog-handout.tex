\documentclass{handout}
\usepackage{amsmath, amssymb}
\usepackage{graphicx}
\usepackage{bm}
\usepackage{hyperref}
\usepackage{subfigure}
\usepackage{amsmath}
\usepackage{amssymb}
\usepackage{wrapfig}
\usepackage{url}
\usepackage{fancyvrb}
\usepackage{pgffor}
\usepackage{minted}
\usepackage{pdfpages}

\SetInstructor{\href{mailto:karl@cs.tufts.edu}{karl@cs.tufts.edu}}
\SetCourseTitle{150-PP}
\SetSemester{Fall 2014}
\SetHandoutTitle{Bayesian Logic (BLOG) Probabilistic Modeling Language}

\begin{document}
\maketitleinst

\section{Modeling}
\begin{itemize}
\item BLOG models specify the structure and contents of \emph{possible worlds}.
\item Define mutually recursive \verb|random| and \verb|fixed| functions.
\end{itemize}

\section{Evidence}
\begin{itemize}
\item Can make arbitrary propositions about the model - propositional conditioning:
\end{itemize}
\begin{minted}{haskell}
obs (x :: Proposition)
\end{minted}

\section{Query}
\begin{itemize}
\item Given any number of observations, can subsequently query the distribution of
      values for a particular function or expression in the model:
\end{itemize}
\begin{minted}{haskell}
query (A (a1, ..., a_n))
\end{minted}

\section{Distribution Representation}
\begin{itemize}
\item Java classes corresponding to specific kinds of distributions
\item Logarithmic probabilities stored in real-valued data type / Java objects
\item Users can `\emph{easily}' add new Java distribution classes
\item Output representation is distribution over arbitrary type, e.g.:
\end{itemize}
\begin{verbatim}
========  LW Trial Stats =========
Log of average likelihood weight (this trial): -3.4266215829983153
Average likelihood weight (this trial): 0.032496542400000314
Fraction of consistent worlds (this trial): 1.0
Fraction of consistent worlds (running avg, all trials): 1.0
======== Query Results =========
Number of samples: 10000
Distribution of values for Weather(@6)
  Dry 0.7168517226620216
  Rainy 0.2831482773379617
Distribution of values for RainyRegion
  true  0.7680060879338363
  false 0.23199391206616182
======== Done ========
\end{verbatim}

\section{Language Strengths}
\begin{itemize}
\item Java EDSL means Java developers can easily interface arbitrary code into the engine
\item Familiar object-oriented style (because of possible worlds), but with necessary functional
      aspects to make models more readable and concise.
\end{itemize}

\section{Language Weaknesses}
\begin{itemize}
\item Modularity
\item Poorly documented \& designed grammar - could have abstracted away a core set of
      probabilistic features and then called out to e.g. a Java grammar specification
      for the general-purpose language constructs
\end{itemize}
\begin{minted}{java}
random Boolean Burglary ~ BooleanDistrib(0.001);

random Boolean Earthquake ~ BooleanDistrib(0.002);

random Boolean Alarm ~
  if Burglary then
    if Earthquake then BooleanDistrib(0.95)
    else  BooleanDistrib(0.94)
  else
    if Earthquake then BooleanDistrib(0.29)
    else BooleanDistrib(0.001);

random Boolean JohnCalls ~
  if Alarm then BooleanDistrib(0.9)
  else BooleanDistrib(0.05);

random Boolean MaryCalls ~
  if Alarm then BooleanDistrib(0.7)
  else BooleanDistrib(0.01);

obs JohnCalls = true;
obs MaryCalls = true;

query Burglary;
\end{minted}

\section{Memorable}
\begin{itemize}
\item Metropolis-Hastings sampling
\item Intuitive runtime system front-end
\end{itemize}

\section{References}
\begin{itemize}
\item BLOG syntax \& semantics - https://bayesianlogic.github.io/download/blog-langref.pdf
\end{itemize}

\end{document}
