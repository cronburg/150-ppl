% Edit this file to produce your paper in LaTeX.

% To format the paper, issue the following commands from the command
% line.
%
% pdflatex template
% pdflatex template
%
% The formatted output will be in template.pdf.  Change the name
% of the formatted file so that it contains your name.


% THIS IS SIGPROC-SP.TEX - VERSION 3.1
% WORKS WITH V3.2SP OF ACM_PROC_ARTICLE-SP.CLS
% APRIL 2009
%
% It is an example file showing how to use the 'acm_proc_article-sp.cls' V3.2SP
% LaTeX2e document class file for Conference Proceedings submissions.
% ----------------------------------------------------------------------------
% This .tex file (and associated .cls V3.2SP) *DOES NOT* produce:
%       1) The Permission Statement
%       2) The Conference (location) Info information
%       3) The Copyright Line with ACM data
%       4) Page numbering
% ----------------------------------------------------------------------------

\documentclass{acm_proc_article-sp}
\usepackage{graphicx}
\usepackage[abbrev]{amsrefs}

% Turn off space for copyright notice.
\makeatletter
\renewcommand{\@copyrightspace}{}
\makeatother

\begin{document}

\title{A Machine Learning Approach to Deck Building Games}

\numberofauthors{1}
\author{
  \alignauthor
  Karl Cronburg                   \\
  \affaddr{Tufts University} \\
  \affaddr{Medford, MA}       \\
  \email{karl@cs.tufts.edu}
}
\date{\today}

\maketitle

\begin{abstract}
\emph{I observe the result of a deck building game where player \#1 wins.
Which cards and how many of each does player \#1 have in her deck,
regardless of what player \#2 has in his deck?}

This kind of question interests those searching for {\bf dominant
strategies} in a particular instance of a deck building game.
Which strategy is dominant relies heavily on the initial configuration
of the game - which cards are chosen. For replayability purposes,
this configuration is chosen from a relatively large space of
possible configurations.

In this project (and subsequent paper) we hope to study the
use of probabilistic programming languages (PPLs) to define generative models
and infer meaning from observations of real-world outcomes as described
by these models. In particular we plan to discuss the desired features
of such languages by comparing the usability of {\bf BLOG} (Bayesian LOGic)
and {\bf Church} in the context of our deck building game inference question.

% Possible separate questions:  
%I see a hand of four coins with one buy.
%I see a hand of five coins and one buy.
%feasibility

\end{abstract}

\section{Introduction}

Deck building games rely both on the underlying mechanics of the game itself,
and the mechanics of the cards in the game. For instance, once a game design team
creating a new deck building game has determined the underlying mechanics of
their game, they can then enrich its playability by designing a variety of cards
to work within this framework.

The existing deck building game we focus on for this project
is {\bf Dominion}.\footnote{by \emph{Donald X. Vaccarino}}
\emph{Dominion} was designed to be replayable. This was done by giving players
$25$ cards to choose $10$ from. As a result there are approximately $3.2$
million variations of the base version of \emph{Dominion}. When the cards from
all versions of \emph{Dominion} are included, we have:

$$\log_{10}{\left[\binom{200 \text{ cards}}{10 \text{ cards}}\right]} = 16.35$$

meaning there are upwards of $10^{16}$ possible game configurations. This
magnitude of the observation state space lends the problem to the use
of machine learning techniques. While a particular instance in this space
could be studied in great detail for dominant strategies, it is a lot more
interesting to model what a dominant strategy looks like in general.

Feeding a dominant strategy model into an existing machine learning algorithm,
we can search for instances of dominant strategies for arbitrary game
configuration instances.

\section{Project Milestones}

The following is a list of goals which, if achieved, would make the project
a success to some extent:

\begin{itemize}
\item Create a model of dominant strategies for \emph{Dominion} in BLOG and / or Church.
\item Learn what makes BLOG and Church good for probabilistic inference, and how they might
      be improved or what support tools are desired.
\item Answer our deck building inference question for a specific instance of \emph{Dominion}
\item Answer our question for various instances of \emph{Dominion}
\item Show that our answer(s) strongly correlate with what domain experts believe.
\end{itemize}

Some of these milestones may or may not be fully attainable in the time of this
course. Further, we have a list of possible future work and goals:

\begin{itemize}
\item Find an instance of a previously undiscovered dominant strategy.
\item Try using other inference systems.
\item Development of domain specific (deck building) tools.
\end{itemize}

\begin{biblist}
\end{biblist}

\balancecolumns

\end{document}
