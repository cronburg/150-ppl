
\section{Introduction} \label{sec:intro}
A {\bf game-play heuristic} is a description of what a player should do to maximize their
utility in a game. Due to the probabilistic nature of many games, this description
can be formulated as a {\bf probabilistic model}. Repeated sampling from this model
gives the modeler a top-down view of what the heuristic does.
The modeler can then, given an inference engine, query the state of the game given
some observations about earlier states of the game.

It is apparent from this description that game-play heuristics fit
into the model-observe-query pattern used by existing Probabilistic Programming
Languages (PPLs).
In this work we present an argument favouring the use of PPLs
for describing and reasoning about game heuristics.

%Our goal is to show that probabilistic inference is directly applicable to game
%heuristics. We argue that the questions a heuristic designer is interested in can be
%reformulated in terms of the model-observe-query pattern used by existing
%PPLs.

\section{Goals} \label{sec:goals}
  
%Existing probabilistic
%The goals of the work presented in the

