
\section{Introduction} \label{sec:intro}
A {\bf game-play heuristic} is a description of what a player should do to maximize their
utility in a game. Due to the probabilistic nature of many games, this description
can be formulated as a {\bf probabilistic model}. Repeated sampling from this model
gives the modeler a top-down view of what the heuristic does.
The modeler can then, given an inference engine, query the state of the game given
some observations about a state of the game at a different point in time.

It is apparent from this description that game-play heuristics fit
into the model-observe-query pattern used by existing Probabilistic Programming
Languages (PPLs).
In this work we present an argument favouring the use of PPLs
for describing game heuristics and reasoning about their performance.

We now continue in Section \ref{sec:goals} with an overview of the goals of this
work as they relate to PPL design. This is followed in Section \ref{sec:moq}
with a discussion of how the probabilistic observe-query model of computation
is applicable to certain questions game designers and players might ask of a
game heuristic. This work then looks at a particular game with sufficiently
complicated probabilistic game mechanics, Dominion, discussed in Section
\ref{sec:dom}.
The complexity of Dominion and our attempts to model and make queries gives
us insight into the desired features of a PPL, and advantages and disadvantages
of existing PPLs. We discuss these insights in Section \ref{sec:meta-analysis}.
This is followed in Section \ref{sec:future:dominion} with a discussion of
Dominion-specific future work. We also touch on in Section \ref{sec:future:PPL}
the future of PPLs in relation to their applicability to the domain of
game heuristics.

%Our goal is to show that probabilistic inference is directly applicable to game
%heuristics. We argue that the questions a heuristic designer is interested in can be
%reformulated in terms of the model-observe-query pattern used by existing
%PPLs.

\section{Goals} \label{sec:goals}
% goals of this work in relation to PPL design

%Existing probabilistic
%The goals of the work presented in the

