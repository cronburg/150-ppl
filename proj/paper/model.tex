
% Probabilistics models, observations, and queries
%\section{Model-Observe-Query} \label{sec:moq}
% Talk about how, in general, game heuristics can be expressively
% described using probablistic models

%\subsection{Example Game}

\subsection{Probabilistic Techniques: Markov Chain Monte-Carlo} \label{sec:mcmc}
% talk about how we use rejection sampling when performing inference
% on probabilistic distributions, and why this is sufficient for
% our purposes >>>>v

% (it's because the points in the state-space which
% evaluate to true in the condition are diffuse and high overall density in the state-space.
% this will generally be true for standard deckbuilding games because
% the uniform-card-shuffling does a robust job of diffusing points which
% evaluate to true across the state space. the uniform shuffling is an inherently
% smooth transformation on the state space. similarly the conditionally true
% points in the state-space are dense because the conditional distributions
% we look at in this paper are not very restrictive - i.e. the percentage of
% conditionally false points in the state-space is very low, making rejection
% sampling successful because it's generally easy to find a conditionally true
% point in the state space regardless of the prior you sample from.

In this work we use rejection sampling to infer distributions over the
latent variables in our model.
In the future we would like to further study instances in which rejection
sampling is insufficient due to high rejection rates. Section
\ref{sec:future} goes into more detail about future work likely requiring
more clever inference techniques.

High rejection rates
in the case of deck building games like Dominion will most likely be the
result of inference questions involving very restrictive conditioning.
While it is possible that a high rejection rate can be the result of
a non-diffuse true distribution, this is not indicative of an interesting
deck of cards. For example you could design a card which turns
of shuffling in the game resulting in a pathologically non-diffuse distribution.
This however is not the kind of card game designers wish to study.

%conditional distributions we are
% currently trying to sample from 

