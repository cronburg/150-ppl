
\section{Future Work} \label{sec:future}

This work touches on the language design aspects of PPLs
in general. We also discussed the applicability of such PPLs
to a specific game (Dominion).

\subsection{Dominion} \label{sec:future:dominion}
%On the latter, future work
%will involve:

This work has demonstrated that modeling Dominion is, by itself,
a difficult task. The modeler must have both an intimate knowledge
of Dominion's mechanics and some level of experience using PPLs.
This is further complicated by the fact that games like Dominion
are described most naturally in a recursive form. Future work should
therefore include a guarantee of correctness of the models presented
in this paper. We are confident in the MCMC techniques used in this
paper. We are however less confident in the exact numeric results
shown in the distributions in the results section. The results
presented do cross-validate with existing beliefs, we should however
perform a more statistically rigorous analysis of these
results.

This future analysis could involve computing values for the
intercepts and slopes of the three distributions in Figure
\ref{fig:rejection-sampling}. We believe this slope $m$ can be used
to compute the relative balance between the cards $X$ and $Y$ in
our greedy model. Namely that balance is proportional to
$1 / abs(m)$. A slope of exactly $m = 0$ would mean the cards
$X$ and $Y$ are perfectly balance (infinite balance). In contrast
a slope approaching $m = \pm \infty$ indicates either $X$ or $Y$
completely dominates the other (perfectly imbalanced). The
conditional distribution we infer would in this case approach
the form of a dirac-delta function:

$$\[P(p_0 | t_e = t_e') \rightarrow \left\{
  \begin{array}{lr}
  \delta(0.0) & : \textrm{ as } m \rightarrow +\infty
  \delta(1.0) & : \textrm{ as } m \rightarrow -\infty
  \end{array}
\right.
\]$$




% TODO: future work discuss automation of analysis

% TODO: discuss fitting algorithms in future work section

% TODO / future work: compare a linear fit of the distribution with
% an exponential one. discuss / think about how we might discuss why
% the underlying game mechanics and our choices for $X$ and $Y$ might
% make either a linear or exponential fit inherently more meaningful
% (e.g. what would the parameters to the exponential model of the
% probability distribution correspond to in terms of balance / interaction
% of the two cards - how can we quantify specific interactions between
% cards. an interaction between two cards on a player's turn in Dominion
% is probabilistically linked to both the total number of turns in the
% game and any latent variables we are interested in

% TODO or future work: compute or approximate values for the intercept
% and slope of the three distributions in the \ref{fig:rejection-sampling}
% figure.

% - the slope $m$ is a relative measure of the balance between cards
%   $X$ and $Y$, namely balance can be interpreted as proportional
%   to $1 / abs(m). A slope of exactly $m = 0$ means the cards $X$
%   and $Y$ are perfectly balanced
% - the intercept $b$ is an absolute measure of how
%   dominant $X$ is compared to $Y$.

% Future work: come up with a formula for the balance of a game in terms
%              of $m$ and $b$


\begin{itemize}
\item Implementation of more Dominion-specific heuristics.
\item Designing heuristics capable of maintaining probabilistic models
      corresponding to belief states. What is the feasability of recursive
      probabilistic models in existing PPLs?
\item Cross-validation with currently held beliefs of human Dominion
      players / domain experts. \ref{human-card-comparisons}
\item Applying Machine Learning algorithms to publicly available Dominion
      data sets. \ref{dominion-data-sets}
\item Design of a probabilistic inference engine to determine
      {\bf balanced} sets of cards to play with.
\end{itemize}

\subsection{PPL Design} \label{sec:future:PPL}
%On the PPL design side, we would like to further explore:

\begin{itemize}
\item Useability of Church, Figaro, probability monad, and other
      PPLs in the context of this work.
\item The utility of a domain specific language for modeling deckbuilding
      games. What do we gain from card-game specific probabilistic
      abstractions?
\end{itemize}

%\section{Conclusions} \label{sec:conc}

\section{Acknowledgements} \label{sec:ack}

