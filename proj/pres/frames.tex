
\begin{frame} \frametitle{Motivation}
\begin{itemize}
\item {\bf Primary Goal: } Want to judge the strength \& limitations of a heuristic
\item Domain experts model `winning` strategies
\item Want to then visualize the probabilistic effects of changing the
      values of parameters in the state space
\item A {\bf probabilistic map} of the state space where the domain expert
      specifies interesting variables to plot distributions over
\end{itemize}
\end{frame}

\begin{frame} \frametitle{Toy Problems}
\begin{itemize}
\item 
\end{frame}

\begin{frame} \frametitle{MCMC Sampling Methods}
\begin{itemize}
\item Rejection Sampling - known model for joint distribution
\item Random Walk
  \begin{itemize}
  \item Metropolis-Hastings - known model for joint distribution
  \item Gibbs Sampling - for hard-to-sample joint distribution
\item 
\end{frame}

\begin{frame} \frametitle{Possible Probabilistic Languages}
\begin{itemize}
\item {\bf Hakaru:} an EDSL in Haskell
  \begin{itemize}
  \item 
  \end{itemize}
\item {\bf BLOG:} a Java-based DSL
  \begin{itemize}
  \end{itemize}
\item Church, Figaro, Fun, others discussed in class...
% Talk about why other languages were not chosen

\begin{frame} \frametitle{Difficulties}
\begin{itemize}
\item Conditioning on recursive models in Hakaru
\item Expressive power of BLOG
\end{itemize}

\begin{frame} \frametitle{Case Study: Deckbuilding Games}
\begin{itemize}
% TODO: reference to Vaccarino from here:
\item We chose \emph{Dominion} - experience
\item {\bf The Process:}
  \begin{itemize}
  % Actual game engine capable of playing a game - probabilities arise
  % from the underlying mechanics of how and when decks are shuffled,
  % and probabilities *can* arise from what cards a player decides to
  % buy or play during their turn.
  \item Write sampling model based on game mechanics \emph{Dominion}
  \item Determine interesting state variables to condition on
  \item Run Hakaru MCMC
  \item Plot resulting joint distributions for domain expert to explore
  \end{itemize}
\end{itemize}
\end{frame}

\begin{frame} \frametitle{Case Study: Dominion Mechanics}
\begin{verbatim}
λ> runGreedy (0.5, 0.5)
Player1:
    name   = "Greedy1"
    hand   = [ESTATE, GOLD, PROVINCE, PROVINCE, SILVER]
    inPlay = []
    deck   = [SILVER,   PROVINCE, COPPER, SILVER,  COPPER
             ,ESTATE,   COPPER,   COPPER, VILLAGE, VILLAGE
             ,PROVINCE, ESTATE,   SILVER, COPPER]
    dscrd  = [SILVER, SILVER, SILVER, COPPER, COPPER, PROVINCE]
    buys=1, actions=1, money=0

Player2:
    name   = "Greedy2"
    hand   = [COPPER, COPPER, COPPER, COPPER, VILLAGE]
    inPlay = []
    deck   = [SILVER,   SILVER, GOLD, COPPER, COPPER,   ESTATE
             ,GOLD,     ESTATE, GOLD, ESTATE, PROVINCE, VILLAGE
             ,PROVINCE, GOLD]
    dscrd  = [SILVER, COPPER, SILVER, SILVER, SILVER, PROVINCE]
    buys=1, actions=1, money=0

Trash: []
Supply: [(COPPER,60),  (CELLAR,10),  (MOAT,10),       (ESTATE,8)
        ,(SILVER,27),  (VILLAGE,6),  (WOODCUTTER,10), (WORKSHOP,10)
        ,(MILITIA,10), (REMODEL,10), (SMITHY,10),     (MARKET,10)
        ,(MINE,10),    (DUCHY,8),    (GOLD,25),       (PROVINCE,0)]
Turn #: 30
\end{verbatim}
\end{frame}

\begin{frame} \frametitle{Case Study: Choice of Parameters}
\begin{itemize}
\item Initially focus on binary choices in card-buying-heuristic
  \begin{itemize}
  \item Buy \verb{VILLAGE} with probability $p0$
  \item Buy \verb{CHANCELLOR} with probability $p1 = 1 - p0$
  \item Condition on number of turns in a game
  \item What does the distribution $P(p0 | turns = t)$ for some $t$ look like?
  \end{itemize}
% TODO: put in FUTURE section - non-binary choices
\end{itemize}
\end{frame}

